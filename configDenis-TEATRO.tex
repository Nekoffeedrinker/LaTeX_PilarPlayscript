%---------------------------------------------------------------------------
%    Configuración general
%---------------------------------------------------------------------------
\usepackage{calc} % Calcular longitudes

%   Variables

% Interlineado
\newcommand{\PsepLine}{1.15}

% Espacio entre párrafos
\newlength{\PsepPar} \setlength{\PsepPar}{6pt}

% Sangría de primer párrafo
\newlength{\PsepIdent} \setlength{\PsepIdent}{0mm}


% Separación de la secciones
\newlength{\PsepLSection} \setlength{\PsepLSection}{0mm}
\newlength{\PsepTSection} \setlength{\PsepTSection}{8pt}
\newlength{\PSepBSection} \setlength{\PSepBSection}{-4pt}

% Separación de las subsecciones
\newlength{\PsepLSubsec} \setlength{\PsepLSubsec}{0mm}
\newlength{\PsepTSubsec} \setlength{\PsepTSubsec}{6pt}
\newlength{\PSepBSubsec} \setlength{\PSepBSubsec}{-4pt}

% Separación de la sub-subsecciones
\newlength{\PsepLSubsub} \setlength{\PsepLSubsub}{0mm}
\newlength{\PsepTSubsub} \setlength{\PsepTSubsub}{6pt}
\newlength{\PSepBSubsub} \setlength{\PSepBSubsub}{-8pt}

% Nombre de la tabla de contenidos
\newcommand{\ToCName}{Escenas}


% Márgenes
\usepackage[letterpaper]{geometry}	% Dimensiones del documento
\geometry{
	left = 40mm,
	right = 40mm,
	top = 25mm,
	bottom = 25mm,
}


%------------------------------------------------
%   Idioma
\usepackage[utf8]{inputenc} % Escribir caracteres latinos
\usepackage[spanish,mexico,es-nodecimaldot]{babel} % Idioma español mexicano

%------------------------------------------------
%   Tipografía

\usepackage[T1]{fontenc}	% Usar fuentes bonitas
\usepackage{notomath}	% Fuente para Serif
\usepackage{courierten}	% Fuente para Mono
\usepackage{atkinson}	% Fuente Sans Serif (anti dislexia)
\renewcommand*\familydefault{\sfdefault} % Activar Sans Serif

\usepackage{sectsty}			% Paquete para usar el comando allsectionsfont
\allsectionsfont{\rmfamily} 	% Establecer los titulares con la fuente Serif

%------------------------------------------------
%   Párrafos

\usepackage{setspace} % Definir separaciones
% En este caso, se utilizó para modificar el interlineado en una parte específica.

\renewcommand{\baselinestretch}{\PsepLine} 	% Interlineado
\setlength{\parskip}{\PsepPar} 		% Espacio entre párrafos
\setlength{\parindent}{\PsepIdent} 	% Sangria de primer párrafo

%   Líneas Viudas o uerfanas
\usepackage[defaultlines=3,all]{nowidow} % Evita las líneas huérfanas

\usepackage{needspace}


%------------------------------------------------
%   listas

\usepackage{enumitem}

\setlist[itemize]{
	topsep=-4pt,      % espacio vertical antes y después del entorno
	itemsep=4pt,     % espacio entre ítems
	parsep=2pt,      % espacio entre párrafos dentro de un ítem
	after=\vspace{4pt} % Espacio después del entorno
}


%------------------------------------------------
%   Titulos

\setcounter{secnumdepth}{0} % Hacer que las secciones no se numeren

\usepackage{titlesec} % Para hacer modificaciones en los titulos
% El ajuste se realiza en {izquierda} {anterior} {posterior}
\titlespacing{\section}{\PsepLSection}{\PsepTSection}{\PSepBSection} % Espaciado de la sección
\titlespacing{\subsection}{\PsepLSubsec}{\PsepTSubsec}{\PSepBSubsec} % Espaciado de la subsección

\usepackage{titlesec}

\titleformat{\subsection}
{\centering\rmfamily\large\bfseries} % Formato del título: centrado, tamaño grande, negritas
{}                       % Número de subsección | Para activarlo: \thesubsection
{1em}                                  % Separación entre número y título
{}                                     % Código antes del título



%------------------------------------------------
%   Tabla de contenidos

% Modificar los títulos cambiados por Babel
\addto\captionsspanish{ 
	\renewcommand{\contentsname}{\begin{center}\rmfamily \ToCName\end{center}}
}

\usepackage{tocloft} % Para personalizar la tabla de contenidos

\renewcommand\cftsecfont{\rmfamily \bfseries \small} % Tamaño de las secciones
\renewcommand\cftsecpagefont{\rmfamily \small} % Tamaño del número de las secciones
\renewcommand\cftsecafterpnum{\vspace{-0.5ex}} % Separación de las secciones

\renewcommand{\cftsecleader}{\cftdotfill{\cftdotsep}} % Añadir puntos

%------------------------------------------------
%   Resaltado y color

\usepackage{xcolor} % Uso de color
\usepackage{soul}	% Hacer resaltados con color

% Comando para resaltar en cualquier color (incluidas combinaciones)
\DeclareRobustCommand{\hlc}[2][yellow]{{%
		\colorlet{foo}{#1}%
		\sethlcolor{foo}\hl{#2}}%
}
% IMPORTANTE, los signos de porcentaje (%) son para evitar espacios indeseados al momento de ejecutar el comando. Por favor, no retirarlos de la definición.

%------------------------------------------------
%   Más utilidades

\usepackage[hidelinks]{hyperref} % Para hacer hipervínculos
\usepackage{ifthen}	% Poder evaluar condicionales

\usepackage{graphicx} 	% Incluir gráficos en el documento
\usepackage{tikz}		% Comandos de dibujo y posicionamiento
\usetikzlibrary{babel,calc} % Librerias para TikZ

