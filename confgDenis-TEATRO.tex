%---------------------------------------------------------------------------
%    Configuración y Paquetes
%---------------------------------------------------------------------------
\usepackage[utf8]{inputenc} % Escribir caracteres latinos
\usepackage[spanish,mexico,es-nodecimaldot]{babel} % Idioma español mexicano

%------------------------------------------------
%   Márgenes

\usepackage[letterpaper]{geometry}	% Dimensiones del documento
\geometry{
	left = 45mm,
	right = 45mm,
	top = 45mm,
	bottom = 45mm,
}

%------------------------------------------------
%   Tipografía

\usepackage[T1]{fontenc} 	% Usar fuentes bonitas
\usepackage{notomath}		% Fuente para Serif y Mono

\usepackage{atkinson}		% Fuente Sans Serif (anti dislexia)
\renewcommand*\familydefault{\sfdefault} % Activar Sans Serif

\usepackage{sectsty}			% Paquete para usar el comando allsectionsfont
\allsectionsfont{\rmfamily} 	% Establecer los titulares con la fuente Serif

%------------------------------------------------
%   Párrafos

\renewcommand{\baselinestretch}{1.2} 	% Interlineado
\setlength{\parskip}{8pt} 		% Espacio entre párrafos
\setlength{\parindent}{0pt} 	% Sangria de primer párrafo

\usepackage[defaultlines=3,all]{nowidow} % Evita las líneas huérfanas

\setcounter{secnumdepth}{0} % Hacer que las secciones no se numeren

%------------------------------------------------
%   Tabla de contenidos (TdC)

\addto\captionsspanish{ % Modificar subtítulos en la TdC, cambiados por Babel
	\renewcommand{\contentsname}{\begin{center}\rmfamily Cuadros\end{center}}
}

\usepackage{tocloft} % Para personalizar la tabla de contenidos

\renewcommand\cftsecfont{\rmfamily \bfseries \small} % Tamaño de las secciones
\renewcommand\cftsecpagefont{\rmfamily \small} % Tamaño del número de las secciones
\renewcommand\cftsecafterpnum{\vspace{-0.5ex}} % Separación de las secciones

\renewcommand{\cftsecleader}{\cftdotfill{\cftdotsep}} % Añadir puntos

%------------------------------------------------
%   Utilidades

\usepackage[hidelinks]{hyperref} % Permite los hipervínculos
\usepackage{ifthen}	% Poder evaluar condicionales

%---------------------------------------------------------------------------
%    Macros y definiciones
%---------------------------------------------------------------------------

%   Evitar saltos de página | NO SE USÓ. REMPLAZADO POR MINIPAGE

% \newenvironment{absolutelynopagebreak}
% {\par\nobreak\vfil\penalty0\vfilneg\vtop\bgroup}
% {\par\xdef\tpd{\the\prevdepth}\egroup\prevdepth=\tpd}

%------------------------------------------------
%   Acotaciones

\newenvironment{acotado}{
	\begin{rmfamily}\begin{itshape}
}{
	\end{itshape}\end{rmfamily}
}

%	Parlamentos

\newenvironment{parla}[2][]{
	\vspace{1em}
	\begin{minipage}{1\linewidth}
		\setlength{\parskip}{8pt} % Espacio entre párrafos
		
		\hspace{25mm} \textrm{\textbf{#2}} \par	\vspace{-8.5pt}
		\ifthenelse{\not\equal{#1}{}}
		{
			\hspace{7mm} \parbox[t]{25em}{(#1)} \\
		}{}
		\begin{habla}
}{
	\end{habla}
	\end{minipage}
	\vspace{0em}
}

\newenvironment{habla}{
	\leftskip=10mm 
	\rightskip=10mm
}{\par}

% Continuación
\newcommand{\cont}{\textsc{\hspace{0.5ex} (cont.)}}

% Resaltar personajes
\newcommand{\personaje}[1]{\emph{\textsf{#1}}}

% Final de un texto
\newcommand{\fintexto}{
	\begin{center}
		\vspace{4em}
		{\large \fontfamily{lmss}\selectfont \bfseries FIN}
	\end{center}
}
